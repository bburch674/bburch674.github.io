\documentstyle[12pt]{report}
%%%%%%%%%%%%
\hoffset=0in
\voffset=0in
%\topmargin -0.5in
\topmargin -1.0in
\headheight 0.0in
\textheight 10.50in 
\footheight 0.0in
\footskip 0.0in
\oddsidemargin -0.0in
\evensidemargin 0.0in 
\textwidth 6.5in

\begin{document}

\pagestyle{empty}

\vspace*{0.5in}

\centerline{\bf STA 575 Practice Problems 1} 

\parindent 0pt

\vspace*{0.5in}

%{\bf 1.} \ \ Using notation developed in class, \\
%{\bf a.} What is the difference between $U_1,...,U_N$ and $y_1,...,y_N$ ? \\
%{\bf b.} Does $P(U_i \ \ \mbox{is selected for the sample})$ depend on the sampling design?  Please explain. \\

{\bf 1.} \ \ What is the difference between sampling error and measurement error?  Please explain. \\

{\bf 2.} \ \ Suppose in SRS wor, $\overline{y}$ is used to estimate $\mu$. \\
{\bf a.} \ \ Show $E(\overline{y}) = \mu$. \\
{\bf b.} \ \ Provide a complete explanation as to what is the meaning of ``expectation." \\
{\bf c.} \ \ Are unbiased estimators better then biased estimators?  
Please explain. \\

{\bf 3.} \ \ Is the following a true statement? ``To invoke the CLT,
the sample size must be large and the finite population should be approximately normally distributed." \\
 
{\bf 4.} \ \ After invoking the CLT, should you use $z_{\alpha/2}$ or
$t_{\alpha/2}$ to construct a CI for a parameter under study? 
Please explain.  \\

{\bf 5.} \ \ Consider a sample size determination problem for $\mu$
in which no previous estimate of $\sigma$ is available.  However,
other studies have come up with estimates of the extreme (min, max) 
values of the variable under study.  Can this information be used in
determining the sample size ($n$)?  Please provide a detailed explanation. \\

{\bf 6.} \ \ Why do exact CIs for population proportions depend on the hypergeometric distribution in SRS wor and the binomial distribution in SRS wr? \\

{\bf 7.} \ \ Consider the scenario in which $\mu_k$, the mean of the $k^{th}$ subpopulation,
is under study.  If a SRS wor of size n is taken, $n_k$ (the number of units selected
from the $k^{th}$ subpopulation) a random variable.  Please describe another sampling design in which $n_k$ is not a random variable. \\

{\bf 8.} \ \ Using sampling with replacement, why is $Var(\widehat{\tau_p}) < Var(\widehat{\tau})$ if the selection probabilities $p_i$ are approximately proportional to the $y_i$-values? \\

{\bf 9.} \ \ {\bf a.} Is $p_i$ in the Hansen-Hurwitz estimator the same as $\pi_i$ in the Horvitz-Thompson estimator?  Explain.  \\
{\bf b.} \ \  If in part {\bf a.} you answered no, under what
circumstances are $p_i$ and $\pi_i$ identical? \\
{\bf c.} \ \ Can the Horvitz-Thompson estimator be used when SRS is performed without replacement? How about with replacement?  Explain in detail how to compute $\pi_i$ in each case. \\

{\bf 10.} \ \ Is the statistical programming language R useful in understanding sampling concepts?  Please explain. 

\end{document}                                                                                                                                                        
